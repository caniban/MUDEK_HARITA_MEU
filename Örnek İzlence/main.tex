\documentclass[12pt,a4paper]{article}
\usepackage[utf8]{inputenc}
\usepackage[margin=2.5cm]{geometry}
\usepackage{graphicx}
\usepackage{titlesec}
\usepackage{array}
\usepackage{hyperref}
\usepackage[colorlinks=true, linkcolor=blue, urlcolor=blue]{hyperref}
\usepackage{color}
\usepackage{float}
\usepackage{fontspec}
\usepackage{longtable}
\usepackage{fancyhdr}
\usepackage{multirow}
\usepackage{booktabs}
\usepackage{sectsty}
\usepackage{setspace}
\usepackage{lipsum}

\setlength{\parindent}{0pt}
\setlength{\parskip}{6pt}
\renewcommand{\baselinestretch}{1.2}

% Section and Subsection font settings
\titleformat{\section}{\large\bfseries\scshape}{}{0em}{}
\titleformat{\subsection}{\normalsize\bfseries}{}{0em}{}
\titlespacing*{\section}{0pt}{3.5ex plus 1ex minus .2ex}{2.3ex plus .2ex}
\titlespacing*{\subsection}{0pt}{3.25ex plus 1ex minus .2ex}{1.5ex plus .2ex}

% Header/Footer Settings
\pagestyle{fancy}
\fancyhf{}
\fancyfoot[C]{\thepage}
\renewcommand{\headrulewidth}{0pt}
\renewcommand{\footrulewidth}{0pt}

% Title Page
\begin{document}
\begin{titlepage}
    \centering
    \begin{minipage}[t]{0.15\textwidth}
        \includegraphics[width=1.5\textwidth]{bölüm_logo.png}
    \end{minipage}
    \hfill
    \begin{minipage}[t]{0.9\textwidth}
        \centering
        \vspace{1cm}
        {\Large \textbf{T.C. MERSİN ÜNİVERSİTESİ}}\\[0.2cm]
        {\Large \textbf{MÜHENDİSLİK FAKÜLTESİ}}\\[0.2cm]
        {\Large \textbf{HARİTA MÜHENDİSLİĞİ BÖLÜMÜ}}\\[1.2cm]
        {\Large \textbf{1606024 COĞRAFİ BİLGİ SİSTEMLERİNİN TASARIMI VE UYGULANMASI DERSİ}}\\[0.5cm]
        {\Large \textbf{2024-2025 AKADEMİK YILI}}\\[0.5cm]
        {\Large \textbf{ÖĞRENİM ÇIKTILARI İLE PROGRAM ÇIKTILARININ DEĞERLENDİRMESİ}}\\[1cm]
        {\large\textbf{DERSİN İZLENCESİ:}}\href{https://drive.google.com/file/d/1orsLVNL4cDUjDvZI44Fa12cOD4x6A5u2/view?usp=drive_link}{\textcolor{blue}{Bağlantı}}
    \end{minipage}
\end{titlepage}

\newpage

\section*{DERS TANITIMI}
Coğrafi Bilgi Sistemlerinin Tasarımı ve Uygulanması dersi, öğrencilere mekânsal veri üretimi, veri tabanı tasarımı, analiz süreçleri ve karar destek haritalarının geliştirilmesi gibi ileri düzey teknik beceriler kazandırmaktadır. Ders kapsamında öğrenciler, Python gibi programlama dilleriyle analiz senaryoları oluşturmakta, ModelBuilder ve SQL gibi araçları kullanarak mekânsal problemleri çözmekte ve AHP gibi yöntemlerle çok kriterli karar destek analizleri gerçekleştirmektedir.

Bu bağlamda, dersin içeriği mühendislik uygulamalarında karşılaşılan karmaşık problemlerin çözümüne yönelik modern teknik ve araçları kullanma becerisi ile doğrudan örtüşmektedir. Öğrenciler bilişim teknolojilerini etkin şekilde kullanarak mühendislik problemlerine çözüm üretmektedir. Bu nedenle ders, "Mühendislik uygulamalarında karşılaşılan karmaşık problemleri çözmek için gerekli olan modern teknik ve araçları seçme ve kullanma becerisi; bilişim teknolojilerini etkin bir şekilde kullanma" çıktısı olan P.Ç.4 ile \%50 oranında örtüşmektedir.

Ayrıca öğrenciler, mekânsal karar destek süreçlerinde veri toplama, analiz etme, yorumlama ve görselleştirme becerilerini geliştirerek mühendislik temelli kararların alınmasına katkı sunmaktadır. Bu yönüyle ders, "Karmaşık mühendislik problemlerinin veya Harita Mühendisliği disiplinine özgü araştırma konularının incelenmesi için deney tasarlama, deney yapma, veri toplama, sonuçları analiz etme ve yorumlama becerisi" çıktısı olan P.Ç.5 ile \%50 oranında örtüşmektedir.

\newpage

\section*{ÖĞRENİM ÇIKTISI 1 (Ö.Ç.1)}
\textbf{Tanım:} CBS veri kaynaklarını analiz eder ve mekânsal karar verme süreçlerine uygun veri setleri oluşturur.\\
\textbf{İlgili Program Çıktısı:} P.Ç.5

\vspace{0.5cm}
\begin{longtable}{p{3cm}p{12cm}}
\toprule
\textbf{Başarı Düzeyi} & \textbf{Açıklama} \\
\midrule
0-50 (Yetersiz) & Öğrenci, CBS veri kaynaklarını tanımlayamamakta ve karar destek süreçleri için uygun veri setlerini oluşturamamaktadır. Veri türleri arasındaki farklara dair bilgi eksikliği bulunmaktadır. \\
\addlinespace
50-60 (İyileştirilebilir) & Öğrenci, bazı veri kaynaklarını tanımakta ancak analiz ve uygun veri seçimi konusunda yetersizdir. Veri formatı, ölçek, güncellik gibi kriterler yeterince dikkate alınmamaktadır. \\
\addlinespace
60-70 (Yeterli) & Öğrenci, temel CBS veri kaynaklarını tanımlayabilmekte ve basit karar destek senaryoları için uygun veri setleri oluşturabilmektedir. Ancak analiz derinliği sınırlıdır. \\
\addlinespace
70-90 (İyi) & Öğrenci, farklı veri kaynaklarını etkili biçimde analiz edebilmekte ve mekânsal karar süreçlerine yönelik anlamlı, uygun veri setleri oluşturabilmektedir. Veri kalitesi ve uygunluğu genellikle doğrudur. \\
\addlinespace
90-100 (Mükemmel) & Öğrenci, veri kaynaklarını kapsamlı biçimde analiz edebilmekte, veri kalitesi, hassasiyet, kapsama alanı gibi kriterleri dikkate alarak karar verme süreçlerine yüksek düzeyde katkı sağlayacak veri setleri üretmektedir. Teknik yeterlilik ve içerik bütünlüğü üst düzeydedir. \\
\bottomrule
\end{longtable}

\vspace{0.5cm}
\textbf{Ölçütler:} 
\begin{itemize}
    \item Ara Sınav: TEST SORU 5,6,11,12,13,14,15,19,20,21
    \item Uygulama: 1, 2, 3, 4, 5, 10
    \item Ödev
    \item Son Sınav: TEST SORU 1,2,3,4,5,23
    \item Makale Öz, Abstract, Giriş, ÇA, Bulgular
\end{itemize}

\textbf{Puanlar:} 
\begin{itemize}
    \item Ara Sınav: 65
    \item Uygulama: 72.5
    \item Ödev: 100
    \item Son Sınav: 56.9 (Makale 86.7)
    \item Ortalama: 76.22
\end{itemize}

\vspace{0.5cm}
\textbf{Değerlendirme:}
Öğrenci, farklı CBS veri kaynaklarını analiz etme ve bu kaynaklardan mekânsal karar süreçlerine uygun anlamlı veri setleri oluşturma konusunda iyi düzeyde performans göstermektedir. Açık kaynak ve kurumsal verileri karşılaştırabilmekte, hangi veri tipinin hangi karar süreçlerine katkı sağlayacağını belirleyebilmektedir. Veri seti oluşturma sürecinde genellikle doğru biçimde seçim yapmakta ve analiz öncesi gerekli dönüşüm işlemlerini uygulayabilmektedir. Ancak bazı durumlarda veri kalitesi değerlendirmeleri sınırlı kalmakta ve karmaşık senaryolarda veri uygunluğu gerekçelendirmesi eksik kalabilmektedir.

\begin{figure}[H]
 \centering
 \includegraphics[width=1.3\textwidth]{cbsUyg Öğrenim Çıktısı 1_Kriterli.png} % Dosya adını güncelledim
\end{figure}

\textbf{İyileştirme Önerileri:}
\begin{itemize}
    \item Veri Kalitesi Değerlendirme Modülleri Eklenmesi
    \item Senaryo Tabanlı Veri Seçimi ve Gerekçelendirme Çalışmaları
    \item Veri Dönüştürme ve Uyumlandırma Atölyeleri
\end{itemize}

\newpage

\section*{ÖĞRENİM ÇIKTISI 2 (Ö.Ç.2)}
\textbf{Tanım:} ModelBuilder, SQL ve network araçları gibi analiz araçlarını kullanarak mekânsal problemleri çözer.\\
\textbf{İlgili Program Çıktısı:} P.Ç.4

\vspace{0.5cm}
\begin{longtable}{p{3cm}p{12cm}}
\toprule
\textbf{Başarı Düzeyi} & \textbf{Açıklama} \\
\midrule
0-50 (Yetersiz) & Öğrenci, ModelBuilder, SQL ya da network analiz araçlarının kullanımına dair temel bilgiye sahip değildir. Analiz senaryolarında bu araçları etkin bir şekilde kullanamamaktadır. \\
\addlinespace
50-60 (İyileştirilebilir) & Öğrenci, bu araçların varlığını ve genel işlevlerini bilmekte, ancak uygulama sırasında hatalı veya eksik analizler yapmaktadır. Problem çözme sürecinde teknik destek gereksinimi yüksektir. \\
\addlinespace
60-70 (Yeterli) & Öğrenci, temel düzeyde ModelBuilder, SQL sorguları ve basit network analizlerini gerçekleştirebilmektedir. Analiz süreçlerinde temel problemleri çözebilecek yeterliliktedir, ancak karmaşık senaryolarda zorlanmaktadır. \\
\addlinespace
70-90 (İyi) & Öğrenci, mekânsal problemleri analiz araçlarını etkin biçimde kullanarak çözebilmektedir. ModelBuilder ile işlem zinciri kurabilir, SQL ile mekânsal sorgular yazabilir ve network analizi ile güzergâh belirleyebilir. Küçük teknik eksiklikler görülebilir. \\
\addlinespace
90-100 (Mükemmel) & Öğrenci, analiz araçlarını bütüncül şekilde kullanarak mekânsal problemleri yüksek doğrulukla çözebilmektedir. Karmaşık senaryoları başarıyla modelleyebilir, süreçleri optimize edebilir ve sonuçları yorumlayabilir. Teknik yeterlilik üst düzeydedir. \\
\bottomrule
\end{longtable}

\vspace{0.5cm}
\textbf{Ölçütler:} 
\begin{itemize}
    \item Ara Sınav: TEST SORU 7,8,9,10,16,17,18
    \item Uygulama: 6, 7, 8
    \item Son Sınav: TEST SORU 6,7,8,9,10,25
    \item Makale Materyal Metod
\end{itemize}

\textbf{Puanlar:} 
\begin{itemize}
    \item Ara Sınav: 77.8
    \item Uygulama: 45.6
    \item Son Sınav: 68.1 (Makale 33.3)
    \item Ortalama: 56.2
\end{itemize}

\vspace{0.5cm}
\textbf{Değerlendirme:}
Öğrenci, ModelBuilder, SQL ve network analiz araçlarının genel işlevlerini kavramış olmakla birlikte, bu araçları uygulamalı olarak kullanma sürecinde zorluk yaşamaktadır. Araçların mantığını açıklayabilse de analiz adımlarını doğru sırayla kurmakta ya da parametreleri doğru biçimde tanımlamakta eksiklikler görülebilmektedir.

\begin{figure}[H]
 \centering
 \includegraphics[width=1.3\textwidth]{cbsUyg Öğrenim Çıktısı 2_Kriterli.png} % Dosya adını güncelledim
\end{figure}

\textbf{İyileştirme Önerileri:}
\begin{itemize}
    \item Adım Adım Uygulamalı Senaryo Çalışmaları
    \item Analiz Sürecine Hata Tespit Modülleri Eklenmeli
    \item Bireysel Uygulama - Akran Geri Bildirimi ile Pekiştirme
\end{itemize}

\newpage

\section*{ÖĞRENİM ÇIKTISI 3 (Ö.Ç.3)}
\textbf{Tanım:} Python programlama diliyle basit CBS analiz senaryoları geliştirir ve uygular.\\
\textbf{İlgili Program Çıktısı:} P.Ç.4

\vspace{0.5cm}
\begin{longtable}{p{3cm}p{12cm}}
\toprule
\textbf{Başarı Düzeyi} & \textbf{Açıklama} \\
\midrule
0-50 (Yetersiz) & Öğrenci, Python diline ve CBS kütüphanelerine aşina değildir. Kod yazmakta veya çalıştırmakta ciddi zorluk yaşamaktadır. Herhangi bir analiz senaryosu geliştirilememektedir. \\
\addlinespace
50-60 (İyileştirilebilir) & Öğrenci, Python'da temel sözdizimine kısmen hâkimdir. Ancak analiz senaryosu oluşturma sürecinde hata oranı yüksektir. Kodlama mantığı ve çıktı üretimi yetersizdir. \\
\addlinespace
60-70 (Yeterli) & Öğrenci, Python kullanarak basit mekânsal analiz senaryoları (örneğin raster hesaplama, nokta verisi işlemleri) geliştirebilmektedir. Ancak kod yapısı ve yorumlama düzeyi sınırlıdır. \\
\addlinespace
70-90 (İyi) & Öğrenci, Python dilini kullanarak temel ve orta düzey CBS analizlerini başarıyla gerçekleştirebilir. Kodlar genellikle hatasız çalışır ve çıktılar analize uygundur. Küçük mantıksal hatalar görülebilir. \\
\addlinespace
90-100 (Mükemmel) & Öğrenci, Python diline ve ilgili CBS kütüphanelerine (ör. geopandas, rasterio, arcpy) üst düzeyde hâkimdir. Analiz senaryoları net, doğru ve yapılandırılmıştır. Kodlar açıklamalı ve optimize edilmiştir. Çıktılar etkin şekilde yorumlanabilir. \\
\bottomrule
\end{longtable}

\vspace{0.5cm}
\textbf{Ölçütler:} 
\begin{itemize}
    \item Son Sınav: TEST SORU 11,12,13,14,15,16,17
\end{itemize}

\textbf{Puanlar:} 
\begin{itemize}
    \item Son Sınav: 61.9
    \item Ortalama: 61.9
\end{itemize}

\vspace{0.5cm}
\textbf{Değerlendirme:}
Öğrenci, Python programlama diliyle basit CBS analiz senaryoları geliştirme konusunda temel yeterliliğe sahiptir. Raster hesaplama, nokta verisi işlemleri gibi giriş düzeyindeki uygulamaları gerçekleştirebilmekte; ancak yazdığı kodlar çoğunlukla örnek temelli ve yeniden üretime dayalıdır.

\begin{figure}[H]
 \centering
 \includegraphics[width=1.3\textwidth]{cbsUyg Öğrenim Çıktısı 3_Kriterli.png} % Dosya adını güncelledim
\end{figure}

\textbf{İyileştirme Önerileri:}
\begin{itemize}
    \item Kod Yapısının Güçlendirilmesine Yönelik Uygulamalar
    \item Kodlama + Yorumlama Bütünlüğünü Geliştirme
    \item Küçük Senaryo Tabanlı Proje Uygulamaları
\end{itemize}

\newpage

\section*{ÖĞRENİM ÇIKTISI 4 (Ö.Ç.4)}
\textbf{Tanım:} AHP yöntemiyle çok kriterli mekânsal analizler gerçekleştirerek karar destek haritaları üretir.\\
\textbf{İlgili Program Çıktısı:} P.Ç.5

\vspace{0.5cm}
\begin{longtable}{p{3cm}p{12cm}}
\toprule
\textbf{Başarı Düzeyi} & \textbf{Açıklama} \\
\midrule
0-50 (Yetersiz) & Öğrenci, AHP yönteminin temel mantığını kavrayamamıştır. Kriter belirleme, ağırlıklandırma ve tutarlılık hesaplamalarında ciddi hatalar yapmakta; karar destek haritası üretememektedir. \\
\addlinespace
50-60 (İyileştirilebilir) & Öğrenci, AHP'nin adımlarına kısmen hâkimdir ancak ağırlık hesaplama ve mekânsal veriyle entegrasyon sürecinde eksikler bulunmaktadır. Üretilen harita yetersiz ya da hatalıdır. \\
\addlinespace
60-70 (Yeterli) & Öğrenci, AHP yöntemiyle temel analiz adımlarını uygulayabilmekte ve basit düzeyde karar destek haritası üretebilmektedir. Ancak tutarlılık ve yorumlama aşamalarında geliştirmeye ihtiyaç vardır. \\
\addlinespace
70-90 (İyi) & Öğrenci, çok kriterli karar verme sürecinde AHP yöntemini doğru şekilde uygulayabilmekte ve tutarlı karar destek haritaları oluşturabilmektedir. Analiz mantığı sağlam, yorumlamalar yerindedir. \\
\addlinespace
90-100 (Mükemmel) & Öğrenci, AHP yöntemine tam anlamıyla hâkimdir. Kriter belirleme, karşılaştırma matrisi oluşturma, ağırlıklandırma, tutarlılık kontrolü ve mekânsal entegrasyon süreçlerini eksiksiz gerçekleştirir. Ürettiği karar destek haritası güçlü, anlamlı ve iyi görselleştirilmiştir. \\
\bottomrule
\end{longtable}

\vspace{0.5cm}
\textbf{Ölçütler:} 
\begin{itemize}
    \item Ara Sınav: TEST SORU 1,2,3,4,22,23,24,25
    \item Uygulama 9
    \item Son Sınav: TEST SORU 18,19,20,21,22,24
    \item Makale AHP, Ağırlıklandırma, AHP Sonucu, Sonuç
\end{itemize}

\textbf{Puanlar:} 
\begin{itemize}
    \item Ara Sınav: Test: 46.9; Uygulama: 0
    \item Son Sınav: 76.4; Makale 60.4
    \item Ortalama: 45.9
\end{itemize}

\vspace{0.5cm}
\textbf{Değerlendirme:}
Öğrenci, AHP (Analitik Hiyerarşi Süreci) yönteminin temel mantığını kavramakta zorlanmaktadır. Kriterlerin belirlenmesi ve bu kriterler arasında yapılan karşılaştırmalar sırasında mantıksal hatalar yapmakta, tutarlılık oranı hesaplamalarında formül ve işlem yanlışlıkları gözlemlenmektedir.
\begin{figure}[H]
 \centering
 \includegraphics[width=1.3\textwidth]{cbsUyg Öğrenim Çıktısı 4_Kriterli.png} % Dosya adını güncelledim
\end{figure}

\textbf{İyileştirme Önerileri:}
\begin{itemize}
    \item Adım Adım Uygulamalı AHP Eğitimi ve Şablon Kullanımı
    \item Basitleştirilmiş Senaryo ile Ağırlıklandırma Egzersizleri
    \item AHP ile Üretilen Haritaların İncelenmesi ve Yorumlama
\end{itemize}

\newpage

\section*{GENEL DEĞERLENDİRME}
Ders kapsamında öğrenciler, AHP (Analitik Hiyerarşi Süreci) yöntemini kullanarak çok kriterli mekânsal analizler gerçekleştirmeyi ve bu analizler doğrultusunda karar destek haritaları üretmeyi hedeflemişlerdir. Genel olarak öğrencilerin yöntemin kavramsal temellerini anlamada zorlandığı; özellikle kriter karşılaştırmaları, ağırlıklandırma matrislerinin oluşturulması ve tutarlılık oranı hesaplamaları gibi temel adımlarda hata yapma eğiliminde oldukları gözlemlenmiştir.

Karar destek haritalarının üretimi sürecinde öğrencilerin teknik araçlara kısmen hâkim olduğu, ancak çok kriterli karar süreçlerinin mantığını sonuç haritasına yansıtmakta zorlandıkları görülmüştür. Bu durum, yöntemin sadece teknik bir hesaplama süreci olarak değil, aynı zamanda mekânsal karar verme bağlamında nasıl kurgulandığını anlamaya yönelik ek destek ihtiyacını ortaya koymaktadır.

Öğrencilerin kavramsal anlayışını güçlendirecek yapılandırılmış örnekler, senaryo temelli uygulamalar ve grup içi karşılaştırma çalışmalarıyla bu becerinin daha etkin şekilde kazandırılabileceği değerlendirilmiştir.

Sonuç olarak, öğrencilerin 4/9'u iyi, 2/9'u yeterli düzeyde öğrenme çıktısını sağlamıştır. Başarısız olan öğrenciler, genellikle dersi düzenli takip etmeyen öğrencilerdir.

\vspace{0.5cm}
\section*{KANITLAR}


Ders Yoklaması:
\href{https://drive.google.com/file/d/1JeWkTLrxBbapeqZN4V_5nImaD7P2ZT2o/view?usp=drive_link}{\textcolor{blue}{Bağlantı}}

Ara Sınav Testi Cevap Anahtarı:
\href{https://drive.google.com/file/d/1wryUOlLb1pAxRwuXdkXKkpBwq-U3oRWJ/view?usp=drive_link}{\textcolor{blue}{Bağlantı}}

Ara Sınav Testi Kağıt Örnekleri:
\href{https://drive.google.com/file/d/1Md2AtZUSq5-wmGOvaASg_6iBitBnzvSu/view?usp=drive_link}{\textcolor{blue}{Bağlantı}}

Ara Sınav Uygulaması Soruları: \href{https://drive.google.com/file/d/1FDyqROkKCLLRm0p6D_WqMCWiMfASUdQx/view?usp=drive_link}{\textcolor{blue}{Bağlantı}}

Ara Sınav Uygulaması Kağıt Örnekleri: \href{https://drive.google.com/file/d/17zi-N6Sfi5DiU6HYQIzmvm-ziamIi86Z/view?usp=drive_link}{\textcolor{blue}{Bağlantı}}

Final Sınavı Cevap Anahtarı: \href{https://drive.google.com/file/d/1RFGc3xPn5SkQCeCQCFbMg2bqLOYANbe0/view?usp=drive_link}{\textcolor{blue}{Bağlantı}}

Final Sınavı Kağıt Örnekleri: \href{https://drive.google.com/file/d/1rPvuzNIJ1FhTq5pxhZo3JQIjdApXDQ9l/view?usp=drive_link}{\textcolor{blue}{Bağlantı}}

Final Sınavı Makale Örnekleri:
\href{https://drive.google.com/file/d/1u5lXlTA9xKJkStcwZ4nQS5OjRsSQxG5T/view?usp=drive_link}{\textcolor{blue}{Bağlantı}}

Final Sınavı Makale Uygulaması Talimatı:
\href{https://drive.google.com/file/d/1OvXfg7KxzEaIX74-8DyeFA1QUzcA-cOr/view?usp=drive_link}{\textcolor{blue}{Bağlantı}}


Soru Bazında Değerlendirme: \href{https://docs.google.com/spreadsheets/d/1J53BQqH7wkT1fnFdQ8SDLvdW3WlpDUPC/edit?usp=drive_link&ouid=113525961747692217101&rtpof=true&sd=true}{\textcolor{blue}{Bağlantı}}

\vspace{0.5cm}
\section*{GENEL BAŞARI DAĞILIMI - }
\begin{center}
\begin{tabular}{lc}
\toprule
\textbf{Puan Aralığı} & \textbf{Öğrenci Sayısı} \\
\midrule
Yetersiz (0-50) & 3 \\
İyileştirilebilir (50-60) & 0 \\
Yeterli (60-70) & 2 \\
İyi (70-90) & 4 \\
Mükemmel (90-100) & 0 \\
\bottomrule
\end{tabular}
\end{center}

\end{document}
